\section{File handling}

\subsection{Abstract file classes}
\begin{tikzpicture}
    \begin{abstractclass}[text width=\umlwidth]{InputFile}{0, 0}
        \attribute{+ info : FileInfo}
        \attribute{+ inforead : Boolean}

        \operation{+ InputFile(path : String)}
        \operation{+ getatoms(frame : Integer)}
        \operation[0]{+ readinfo() : FileInfo}
        \operation[0]{+ readatoms(frame : Integer) : Atoms}
    \end{abstractclass}

    \begin{abstractclass}[text width=\umlwidth]{ResultFile}{0, -6}
        \inherit{InputFile}

        \attribute{+ info : ResultInfo}

        \operation{+ ResultFile(path : String, sourcefilepath : String)}
        \operation{+ getresults(frame : Integer, resolution : Integer)}
        \operation{+ addresults(results : Results)}
        \operation[0]{+ readinfo() : ResultInfo}
        \operation[0]{+ writeinfo()}
        \operation[0]{+ readresults(frame : Integer, resolution : Integer) : Results}
        \operation[0]{+ writeresults(results : Results)}
    \end{abstractclass}
\end{tikzpicture}

\todo{Maybe \texttt{ResultFile.writeresults} should take an \texttt{overwrite} parameter.}

\noindent
The Implementation of \texttt{Inputfile} is \texttt{XYZFile}, which reads a pybel xyz file. \texttt{HDF5File} implements \texttt{ResultFile} and handles a HDF5 file with input and result data, which can be used for caching.


\subsection{Filesystem access}
\begin{tikzpicture}
    \begin{class}[text width=\umlwidth]{FileManager}{0, 0}
        \attribute{+ directory : String}
        \attribute{+ types : Dictionary(String $\rightarrow$ Class)}
        
        \operation{FileManager(directory : String)}
        \operation{abspath(filename : String) : String}
        \operation{filelist() : List(String)}
        \operation{\_\_getitem\_\_(filename : String) : InputFile}
        \operation{\_\_contains\_\_(filename : String) : Boolean}
    \end{class}
\end{tikzpicture}

The \texttt{types} dictionary is used to associate filename extensions (e.\,g.\ xyz) with implementatios of \texttt{InputFile}.
\texttt{\_\_getitem\_\_} looks up which class to use and instanciates it.
