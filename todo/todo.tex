\thispagestyle{empty}
\begin{itemize}
\item zentralen Eventhandler ausprogrammieren
\item view-Tab, saved views
\item image/video-Tab,
\begin{itemize}
	\item Videoaufnahme-Funktion
	\item Speichern von Benutzereingaben
	\item Screenshot Funktion
\end{itemize}
\item Konsolen Modus
\item mehr Interaktion mit dem OpenGL window
\begin{itemize}
	\item Translation
	\item Auswahl von Atomen
	\item Zentrieren der Ansicht um bestimmte Objekte
\end{itemize}
\item leerer Startbildschirm ohne geladenen Datensatz
\item statistics-Tab
\begin{itemize}
	\item Entwickeln einer TreebookCtrl
	\item Darstellen der Daten des geladenen Datensatzes
	\item Interaktion mit dem OpenGL window
	\item ggf. Referenzen zwischen den einzelnen Objekten mittels Hyperlink
\end{itemize}
\item Menu-Struktur für bestimmte Funktionen (Recently used datasets, etc.)
\item export-Funktionen
\item GUI Hänger durch Global Interpreter Lock beheben, ggf. \textit{multiprocessing}
\item Erweitern der Konfigurationsdatei
\begin{itemize}
	\item OpenGL view-settings
	\item recent files
\end{itemize}
\item Berechnung der dihedral angles
\item Bondberechnung
\item sicheres Abbrechen der Berechnung ermöglichen
\item tuple support für den Config-Parser
\item Hilfe, Tooltips, Übersetzung
\end{itemize}
\clearpage

\thispagestyle{empty}
\section*{Refactor-TODO pyMolDyn 2}
\begin{itemize}
	\item Berechnungs- und Abfrage aus der GUI entfernen
	\item Objekt zum Speichern der Berechnungseinstellungen
	\item zentrales Callbackobjekt
	\item Framechoose und CalculationTable geordnet trennen
	\item ggf. Kamera-Mechanismus ändern
\end{itemize}